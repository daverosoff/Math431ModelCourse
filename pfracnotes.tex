\def\encoding{UTF-8}
\input{mmd-article-header}
\def\mytitle{Partial fractions expansions}
\def\subtitle{Review and further development for Laurent series and residues}
\def\affiliation{The College of Idaho}
\def\mydate{1 April 2013}
\def\latexmode{article}
\def\fonttheme{structurebold}
\def\colortheme{crane}
\def\theme{Szeged}
\input{mmd-article-begin-doc}
Let us find the partial fractions decomposition of
 \[
R(z) = \frac{z^2 + 4}{(z-2)(z-3)^2}.
\] 
The denominator is already factored and its degree exceeds that of the numerator. The PF expansion takes the form indicated in Theorem 2 of section 3.1:
 \[
\frac{z^2 + 4}{(z-2)(z-3)^2} = \frac{A^{(1)}_{0}}{z-2} + \frac{A^{(2)}_{0}}{(z - 3)^2} + \frac{A^{(2)}_{1}}{z - 3}
\]
 
Multiplying by $ z - 2 $ and taking a limit, we find that
 \[
A^{(1)}_{0} = \lim_{z \to 2} (z - 2) R(z) = 8.
\] 
To find $ A^{2}_{0} $ the process is similar:
 \[
A^{(2)}_0 = \lim_{z \to 3} (z - 3)^2 R(z) = 13.
\] 
Finally, to find $ A^{(2)}_{1} $, we add another step of differentiating. Observe that the $ z $-derivative of $ (z-3)^2 R(z) $ is
 \[
\frac{d}{dz} \left( (z-3)^2 R(z) \right) = \frac{d}{dz} \left( \frac{A^{(1)}_{0} (z-3)^2}{z-2} \right) + A^{(2)}_{1}}.
\] 
Taking the limit of each side as $ z \to 3 $, we find that $ A^{(2)}_{1} $ is equal to
 \[
\lim_{z \to 3} \frac{2z(z-2) - (z^2 + 4)}{(z-2)^2} = \frac{2 \cdot 3 \cdot 1 - 13}{1^2} = -7.
\] 

\begin{itemize}
\item this is a rigged example, often the coefficients are fractional

\item don't be alarmed if coefficients are sometimes 0

\item analogy to factored form for zeroes.

\item how do zeroes and poles behave under differentiation

\item problem 4

\item residues

\item problems 19, 20, 21

\end{itemize}

\input{mmd-memoir-footer}

\end{document}
