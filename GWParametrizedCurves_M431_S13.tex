\documentclass[11pt]{exam}
% \usepackage{pslatex}
\usepackage{xparse}
\usepackage{graphicx}
\DeclareGraphicsExtensions{.jpg, .png}
\usepackage{amsmath}
\usepackage{fourier}
%\usepackage{amsfonts}
\usepackage{enumerate}
\usepackage{siunitx}
\firstpageheader{}{}{}
\runningheader{\textbf{Spring 2013}}
 {}
 {\textbf{Math 431}}
 %{\emph{Page \thepage~of \numpages}}
\runningheadrule
\setlength{\parskip}{1ex}
\setlength{\parindent}{0pt}
\pagestyle{head}
%\newcommand{\N}{\mathbb{N}}
%\newcommand{\Z}{\mathbb{Z}}
%\newcommand{\R}{\mathbb{R}}
%\newcommand{\dwrspace}[1]{\vspace*{\stretch{#1}}}
\NewDocumentCommand\N{}{\mathbf{N}}
\NewDocumentCommand\R{}{\mathbf{R}}
\NewDocumentCommand\Z{}{\mathbf{Z}}
\NewDocumentCommand\Q{}{\mathbf{Q}}
\NewDocumentCommand\dwrspace{m}{\vspace*{\stretch{#1}}}
\begin{document}
\noindent
\textbf{{\large Mathematics 431 \\ Parametrized Curves and Line Integrals}}
% \hfill Name: \underline{\hspace{0.5in}Answers\hspace{2in}}

\noindent
April 12, 2013 \hfill Name: \underline{\hspace{3in}} 

\noindent
Due: April 15, 2013

\noindent
\begin{figure}[h]
\centering
\begin{minipage}[b]{0.85\linewidth}
\textbf{Introduction.} In this activity we'll remind ourselves how to parametrize simple curves. In complex analysis, the most important curves to understand are arcs of circles and line segments.
\end{minipage}
\end{figure}

Recall that the unit circle is the set of complex numbers of magnitude 1. These numbers, as by now should be completely familiar, are all of the form $e^{i \theta}$. If we think of $\theta$ as taking values from the interval $[0, 2\pi]$, then the unit circle becomes a simple closed curve. It is \emph{parametrized} by the interval, because each point in the interval gives a point on the circle. That is all the word parametrization means: putting things into correspondence with numbers.

\begin{questions}

\question Parametrize the circle of radius 4 centered at 0.

\dwrspace{1}

\question What is the parameter interval for your circle in the previous problem? Reparametrize so that your parameter takes only values in the interval $[0,1]$ (the unit interval).

\dwrspace{1}

\question Give a parametrization of the circle of radius $3/8$ centered at $-1/2 + i \sqrt{3}/2$ with parameter interval $[0,1]$.

\dwrspace{1}

\question Give a parametrization of the unit circle, but starting at $z = -1$ instead of $z = 1$.

\dwrspace{1}

\question Give a parametrization of the unit circle, but going backwards instead of forwards (clockwise instead of anticlockwise).

\newpage

\question Parametrize the line segment starting at 0 and ending at $i$. What is the parameter interval?

\dwrspace{1}

\question Parametrize the line segment starting at 1 and ending at $1+2i$. Can you find a parametrization using the interval $[0,1]$?

\dwrspace{1}

\question Parametrize the line segment with endpoints $-i$ and $3 + 4i$, in both directions. 

\dwrspace{1}

\question The \emph{length} of a parametrized curve $z = z(t)$, $a \leq t \leq b$ is by definition
\[
    \int_a^b \left| \frac{dz}{dt} \right| \; dt.
\]
Find the lengths of the segments you have parametrized, using this formula.

\dwrspace{1}

\question Now let $\alpha$ be any complex number. Find a parametrization of the circle of radius $\rho > 0$, centered at $\alpha$, such that $|dz/dt| = 1$. Such a parametrization is called an \emph{arc length parametrization}. 

\dwrspace{1}

\question Use your parametrization and the length formula to show that the circumference of the circle is $2 \pi \rho$.
\end{questions}

\end{document}