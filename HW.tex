\def\encoding{UTF-8}
\input{mmd-article-header}
\def\mytitle{Math 431 Homework and Quiz Schedule}
\def\myauthor{Dave Rosoff}
\def\mydate{February 6, 2013}
\def\htmlheaderlevel{1}
\def\latexmode{article}
\input{mmd-article-begin-doc}
\chapter{Math 431: Complex Analysis}
\label{math431:complexanalysis}

\href{LectureNotes.html}{Visit the lecture notes}\footnote{\href{LectureNotes.html}{LectureNotes.html}}

\subsection{Daily homework, reading, quiz dates}
\label{dailyhomework}

Looking for the weekly homework (\autoref{weeklyhomework})?

These problems are not to be turned in, but all quiz problems will come verbatim from daily problem sets. It's a good idea to keep up with them as best you can. Some of them are repetitive. I trust you to use your discretion and work as many problems as you need to, until you understand what's happening.

\subsubsection{Daily problems: Week 12}
\label{dailyproblems:week12}

\begin{itemize}
\item Friday, May 10: The Riemann Mapping Theorem and covering spaces (?); loose ends.

\item Wednesday, May 8: Conformal mapping and the Dirichlet problem. M\"obius transformations.

\item Monday, May 6: More heat transfer. Introduction to conformal mapping.

\end{itemize}

\subsubsection{Daily problems: Week 11}
\label{dailyproblems:week11}

\begin{itemize}
\item Friday, May 3: The Cauchy estimates. Seriously cool consequences. Applications to heat transfer.

\begin{itemize}
\item Section 4.6: 1, 8, 10.

\end{itemize}

\item Wednesday, May 1: Cauchy's Integral Formula

\begin{itemize}
\item Section 4.5: 1, 3, 4, 7.

\end{itemize}

\item Monday, April 29: Cauchy's Integral Theorem--``deformations'' approach.

\begin{itemize}
\item Section 4.4 (quiz 5\slash 1): 1, 2 (pictures encouraged!), 12.

\end{itemize}

\end{itemize}

\subsubsection{Daily problems: Week 10}
\label{dailyproblems:week10}

\begin{itemize}
\item Friday, April 26: Exam 3: sections 3.1--3.4, 4.1--4.2.

\item Wednesday, April 24: Path independence and homotopy. Quiz on 4.1--4.2.

\begin{itemize}
\item Section 4.3 (quiz 5\slash 1): 1, 7, 12 (use the ML-estimate).

\end{itemize}

\item Monday, April 22: Class cancelled.

\end{itemize}

\subsubsection{Daily problems: Week 9}
\label{dailyproblems:week9}

\begin{itemize}
\item Friday, April 19. Contour integrals.

\end{itemize}

 - Friday, April 19 (no quiz before exam 3): Applications: heat transfer and harmonic functions. 
    - Section 3.4: 1, 3.
    - Section 3.5: 1--4. 

\begin{itemize}
\item Wednesday, April 17 (quiz 4\slash 22): Contour integrals.

\begin{itemize}
\item Section 4.2: 3--11 odd.

\end{itemize}

\item Monday, April 15 (quiz 4\slash 22): Contours and contour integrals. Know how to parametrize line segments and arcs of circles.

\begin{itemize}
\item Section 4.1: 1--5.

\end{itemize}

\end{itemize}

\subsubsection{Daily problems: Week 8}
\label{dailyproblems:week8}

\begin{itemize}
\item Friday, April 12 (quiz 4\slash 17): More complex logarithms.

\begin{itemize}
\item Section 3.3: 1--4, 12, 13.

\end{itemize}

\item Wednesday, April 10 (quiz 4\slash 17): The complex logarithm.

\item Monday, April 8 (quiz 4\slash 12): Elementary functions: $\sin$, $ \cos $, and $ \exp $. Read section 3.2 and skim 3.3.

\begin{itemize}
\item Section 3.2: 1--6, 9--11.

\end{itemize}

\end{itemize}

\subsubsection{Daily problems: Week 7}
\label{dailyproblems:week7}

\begin{itemize}
\item Friday, April 5: How to compute residues. Skim section 3.2.

\item Wednesday, April 3: More on rational functions, poles and zeroes on $\widehat{\mathbf{C}}$.

\item Monday, April 1: Conclude 3.1. Rational functions and poles.

\begin{itemize}
\item Section 3.1 (quiz 4\slash 5 4\slash 8): 8, 11, 13, 15.

\end{itemize}

\end{itemize}

\subsubsection{Daily problems: Week 6}
\label{dailyproblems:week6}

\begin{itemize}
\item Friday, March 22: Not much happened.

\begin{itemize}
\item Section 3.1: 1--3.

\end{itemize}

\item Wednesday, March 20: Exam 2. Reread 3.1 for Friday.

\item Monday, March 18: No daily problems; review for Exam 2.

\end{itemize}

\subsubsection{Daily problems: Week 5}
\label{dailyproblems:week5}

\begin{itemize}
\item Friday, March 15: Read 3.1.

\item Wednesday, March 13: Read 2.7 and skim 3.1. The best links so far that I have found for Julia sets and the Mandelbrot set are \href{http://classes.yale.edu/fractals/Mandelset/welcome.html}{here}\footnote{\href{http://classes.yale.edu/fractals/Mandelset/welcome.html}{http:/\slash classes.yale.edu\slash fractals\slash Mandelset\slash welcome.html}} and \href{http://en.wikibooks.org/wiki/Pictures_of_Julia_and_Mandelbrot_Sets/The_Mandelbrot_set}{here}\footnote{\href{http://en.wikibooks.org/wiki/Pictures_of_Julia_and_Mandelbrot_Sets/The_Mandelbrot_set}{http:/\slash en.wikibooks.org\slash wiki\slash Pictures\_of\_Julia\_and\_Mandelbrot\_Sets\slash The\_Mandelbrot\_set}}.

\begin{itemize}
\item Section 2.6: No daily exercises.

\item Section 2.7: 1, 3a.

\end{itemize}

\item Monday, March 11 (no quiz before Exam 2): Read section 2.5 and skim 2.6, 2.7.

\begin{itemize}
\item Section 2.5: 2, 5, 8.

\end{itemize}

\end{itemize}

\subsubsection{Daily problems: Week 4}
\label{dailyproblems:week4}

\begin{itemize}
\item Friday, March 8 (quiz 3\slash 13): Read section 2.4 and skim 2.5.

\begin{itemize}
\item Section 2.4: 2, 5, 7.

\end{itemize}

\item Wednesday, March 6 (quiz 3\slash 13): Read section 2.3 and skim 2.4.

\begin{itemize}
\item Section 2.2: 1--3, 11.

\item Section 2.3: 7, 9, 11.

\end{itemize}

\item Monday, March 4 (quiz 3\slash 8): Read section 2.2 and skim 2.3. No new problems.

\end{itemize}

\subsubsection{Daily problems: Week 3}
\label{dailyproblems:week3}

\begin{itemize}
\item Friday, March 1 (quiz 3\slash 8): Read section 2.1 and skim section 2.2.

\begin{itemize}
\item Section 2.1: 1--4, 7--9.

\end{itemize}

\item Wednesday, February 27: Exam 1.

\item Monday, February 25: No new problems.

\end{itemize}

\subsubsection{Daily problems: Week 2}
\label{dailyproblems:week2}

\begin{itemize}
\item Friday, February 22 (quiz 3\slash 1). Read section 2.1 and skim section 2.2.

\begin{itemize}
\item Section 2.1: 1--4, 7--9.

\end{itemize}

\item Wednesday, February 20 (quiz 2\slash 25). Read section 1.7 and skim section 2.1.

\begin{itemize}
\item Section 1.7: 1--4.

\end{itemize}

\item Monday, February 18 (quiz 2\slash 25). Read sections 1.5 and 1.6, and skim 1.7.

\begin{itemize}
\item Section 1.5: 4, 5, 13.

\item Section 1.6: 1--10.

\end{itemize}

\end{itemize}

\subsubsection{Daily problems: Week 1}
\label{dailyproblems:week1}

\begin{itemize}
\item Friday, February 15 (quiz 2\slash 20). Read section 1.4 and skim 1.5.

\begin{itemize}
\item Section 1.4: 1, 4, 5, 7, 12, 17.

\end{itemize}

\item Wednesday, February 13 (quiz 2\slash 20). Read Section 1.3 and skim 1.4.

\begin{itemize}
\item Section 1.3: problems 1, 2, 3, 5, 6, 7, 9.

\end{itemize}

\item Monday, February 11 (quiz 2\slash 15). Read Sections 1.1--1.3.

\begin{itemize}
\item Section 1.1: problems 1, 3, 4, 5--13, 14, 18, 19, 23.

\item Section 1.2: problems 1, 2, 3, 4, 7, 9.

\end{itemize}

\end{itemize}

\subsection{Weekly homework}
\label{weeklyhomework}

Looking for the daily homework (\autoref{dailyhomework})?

This is a senior-level mathematics course, and your homework should reflect the high level of professionalism you have developed in your studies. It should be clear, concise, readable, and grammatically correct; it should be made of paragraphs that are themselves made of sentences; and it should be convincing. If there is a gap in your argument, don't try to hide it. Point it out and indicate exactly what is missing. Unsatisfactory write-ups earn no credit, no matter how obvious the underlying reasoning might appear to you.

If you are interested in typing your homework, I encourage you to learn the typesetting system $\mathrm{\LaTeX}$. It is a valuable skill for people who plan to do any amount of scientific writing, and it is much easier today than it once was. I have loved using it for an embarrassingly long time and am happy to help you. See me if you are interested in getting started; you can learn most of what you'll need for homework sets in an afternoon (it's fun). All my syllabi, quizzes, exams, and other course handouts are typeset in $\mathrm{\LaTeX}$, as are most professional mathematical books and articles (including graduate dissertations---at least, the dissertations of people who finish on time and do not have to hire Word professionals).

\subsubsection{Problem Set 9: Sections 4.3--4.6  due Monday, May 6.}
\label{problemset9:sections4.3--4.6brduemondaymay6.}

\begin{itemize}
\item Section 4.4: 4, 10.

\item Section 4.5: 2, 6, 8.

\item Section 4.6: 5, 6,

\end{itemize}

\subsubsection{Problem Set 8: Sections 4.1--4.2  due Monday, April 29.}
\label{problemset8:sections4.1--4.2brduemondayapril29.}

\begin{itemize}
\item Section 4.1: 8, 9, 13.

\item Section 4.2: 6, 10, 12.

\item Section 4.3: 2, 3 (hint: use the ML-estimate), 5.

\end{itemize}

\subsubsection{Problem Set 7: Sections 3.3--3.5  due Monday, April 22.}
\label{problemset7:sections3.3--3.5brduemondayapril22.}

\begin{itemize}
\item Section 3.3: 5, 11, 15, try 19.

\item Section 3.4: 5, 6.

\item Section 3.5: 5, 6, 7.

\end{itemize}

\subsubsection{Problem Set 6: Sections 3.1--3.2  due Friday, April 12}
\label{problemset6:sections3.1--3.2brduefridayapril12}

\begin{itemize}
\item Section 3.1: 7, 14.

\item Section 3.2: 19, 20, 23.

\end{itemize}

\subsubsection{Problem Set 5: Sections 2.3--2.7}
\label{problemset5:sections2.3--2.7}

\begin{itemize}
\item Section 2.3: 16.

\item Section 2.4: 7, 8. Also read and think about 9--14. These kinds of results have wider applicability than you might guess.

\item Section 2.5: 13. (You will probably need the result of 12, but the proof of 12 is not hard.)

\item Section 2.6: 1--3. The pictures are enough (you don't need to discuss them at any length) but I want you to really think about them. Draw \emph{big}, \emph{beautiful} pictures for this problem.

\item Section 2.7: 2. (Ape the solution of Example 1.)

\end{itemize}

\subsubsection{Problem Set 4: Sections 2.1--2.2}
\label{problemset4:sections2.1--2.2}

\begin{itemize}
\item Section 2.1: 12, 13; Bonus: 14, 15.

\item Section 2.2: 9, 12, 25.

\end{itemize}

\subsubsection{Problem Set 3: Sections 1.6--1.7}
\label{problemset3:sections1.6--1.7}

\begin{itemize}
\item Section 1.6: 11, 12, 16; Bonus: 17.

\item Section 1.7: 6, 8, 9.

\end{itemize}

\subsubsection{Problem Set 2: Sections 1.4--1.5}
\label{problemset2:sections1.4--1.5}

\begin{itemize}
\item Section 1.4: 11 (with proofs or counterexamples), 18, 20, 22, 23.

\item Section 1.5: 1, 8, 10.

\end{itemize}

\subsubsection{Problem Set 1: Sections 1.1--1.3}
\label{problemset1:sections1.1--1.3}

\begin{itemize}
\item Section 1.1: 15--17, 24, 25; Bonus: 30.

\item Section 1.2: 6, 8, 10, 11.

\item Section 1.3: 10, 11, 22; Bonus: 28.

\end{itemize}

On your first weekly grade report, the problems correspond as follows: 1.1.17; 1.1.24; (1.1.30); 1.2.8; 1.3.22; (1.3.28); 1.4.11; 1.5.10. Bonus problems appear in parentheses. They can contribute to the numerator when I calculate your homework score, but they contribute 0 to the denominator.

\input{mmd-memoir-footer}

\end{document}
